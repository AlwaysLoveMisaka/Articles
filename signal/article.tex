\documentclass[UTF8,a4paper,11pt]{article}

\usepackage{ctex} 
\usepackage{fancyhdr}
\usepackage{multicol} 
\usepackage{lastpage} 
\usepackage{geometry} 
\usepackage{titlesec} 
\usepackage{graphicx}
\usepackage{epstopdf}
\usepackage{ulem}
\usepackage{amsmath}
\usepackage[subfigure,AllowH]{graphfig} 

\geometry{left=3cm,right=3cm,top=2.5cm,bottom=2.5cm}
\renewcommand{\baselinestretch}{1.5}


\title{\huge\heiti《信号与系统》(刘泉、江雪梅主编)复习手册}
\author{AlwaysLoveMisaka}
\date{\today}

\begin{document}
\maketitle
\thispagestyle{empty}
\clearpage

\tableofcontents
\thispagestyle{empty}
\clearpage

\setcounter{page}{1}
\section{前言}
这篇文章是我写的一篇用于复习《信号与系统》的手册,也是我用\LaTeX 写的第一篇文章。通过这篇文章,我打算梳理《信号与系统》(刘泉、江雪梅编)的重点知识。同志们在使用这个手册的时候,最好要结合课本,我也会在文中提及课本第一版(别问我为什么不用最新版,问就是没钱)的内容。

这个手册是在Github上免费提供的,以后也会同步到CSDN上,如果同志们觉得这个好用,可以给我的内容点一个赞。本人Github主页:https://github.com/AlwaysLoveMisaka,CSDN主页:https://blog.csdn.net/Communism1848?type=blog。

需要指明的是,本手册并不是考纲。所以,同志们如果发现手册里没有一些考试的内容,请不要来打我,毕竟我就是个垃圾大学生,不是出卷老师。没有写入的内容,部分是我认为不重要或者大家肯定懂,部分是我也没弄明白,部分是我技术有限,很难用\LaTeX 写出来(比如框图,我作为一个\LaTeX 新手是真的不会画)。所以请同志们见谅。

这个手册可以写出来,我首先要感谢的是刘泉、江雪梅两位老师,她们既是教材的编者,也是《信号与系统》这门课我的授课老师。我认为这种情况对于我个人是非常难得的,没有她们对教材的深入讲解,我估计写不出来什么东西。我还要感谢我的父母,他们给我的大学生活提供了经济支持。

另外,我还要感谢御坂美琴,虽然她是一个虚拟的人物,但却是我真实的\sout{老婆}精神支柱。每当想起她,我的生活就充满动力。我好想做御坂大人的狗啊,可是御坂大人说她喜欢的是猫(她是真喜欢猫,如图1),我哭了。我知道既不是狗也不是猫的我为什么要哭的,因为我其实是一只老鼠。我从没奢望御坂大人能喜欢自己,我明白的,所有人都喜欢萌萌的狗狗或者猫猫,没有人会喜欢阴湿带病的老鼠。(此段为作者逆天发病,若感到不适,可以用记号笔涂黑)
\newline
\begin{figure}[htbp]
\centering
\includegraphics[scale=0.2]{cat.jpg}
\caption{御坂妹妹和猫}
\label{figure}
\end{figure}

\section{信号与系统的基本概念}
\subsection{信号的分类}
连续时间信号 $f(t)$ 的能量 $E$ 和功率 $P$ 分别定义为
\begin{equation}
E=\lim_{T \to \infty}\int_{-T}^{T} {f^2(t)} \,{\rm d}t
\end{equation}
\begin{equation}
P=\lim_{T \to \infty}\frac{1}{2T}\int_{-T}^{T} {f^2(t)} \,{\rm d}t 
\end{equation}

连续时间信号 $f(k)$ 的能量 $E$ 和功率 $P$ 分别定义为
\begin{equation}
E=\sum_{k=-\infty}^\infty f^2(k)
\end{equation}
\begin{equation}
P=\lim_{N \to \infty}\frac{1}{2N}\sum_{k=-N}^N f^2(k)
\end{equation}

若 $0<E<\infty$ ,且 $P=0$ ,则称此信号为能量信号,即信号能量有限。若 $0<P<\infty$ ,且 $E$ 趋近于无穷,则称此信号为功率信号,即信号功率有限。

从以上叙述中,我们可以知道,信号可以分成三类:能量信号、功率信号、既不是能量信号也不是功率信号的信号(如 $t\epsilon(t)$ )。一般来说,周期信号都是功率信号。

\subsection{系统的分类}
从数学角度来说,系统可定义为实现某功能的运算。设符号 $T$ 表示系统的运算,将输入信号(又称激励) $e(t)$ 作用于系统,得到输出信号(又称响应) $r(t)$ ,表示为
\begin{equation}
r(t)=T[e(t)] 
\end{equation}

此关系也经常用 $e(t)\to r(t)$来表示。

线性系统是指满足齐次性和叠加性的系统,可以用符号描述为,若 $e_1(t)\to r_1(t)$ , $e_2(t)\to r_2(t)$ ,则
\begin{equation}
T[k_1e_1(t)+k_2e_2(t)]=k_1T[e_1(t)]+k_2T[e_2(t)] 
\end{equation}

其中 $k_1$ 、 $k_2$ 是常数。若不满足上述条件,则为非线性系统。

时不变系统的特性如下
\begin{equation}
r(t-\tau)=T[e(t-\tau)]
\end{equation}

其中 $\tau$ 是常数。若不满足上述条件,则为时变系统。

因果系统是指系统在 $t_0$时刻的响应只取决于 $t\leq t_0$ 时的输入,否则为非因果系统。一般而言,任何物理可实现系统都具有因果性。而理想系统,例如各类理想滤波器,往往具有非因果性。

稳定系统可描述为:若 $\left | e(t) \right |\leq M_1<\infty$ ,则$\left | r(t) \right |\leq M_2<\infty$ 。如不满足上述条件,则为不稳定系统。

本教材主要讨论的是线性时不变系统。

\section{连续时间信号与系统的时域分析}
\subsection{常用典型信号}
抽样信号 $Sa(t)$ 的定义为
\begin{equation}
Sa(t)=\frac{sint}{t}
\end{equation}

不难证明
\begin{equation}
\int_{-\infty}^{+\infty} {Sa(t)} \,{\rm d}t=\pi
\end{equation}

单位阶跃信号 $\epsilon(t)$ 的定义为
\begin{equation}
\epsilon(t)=
\begin{cases}
1\quad t>0 \\
0\quad t<0 \\
\end{cases}
\label{pythagorean}
\end{equation}

值得注意的是, $t=0$ 时, $\epsilon(t)$ 未定义。

我们可以用单位阶跃信号表示一些特殊信号,如矩形脉冲信号 $G_{\tau}(t)$ ,它可表示为
\begin{equation}
G_{\tau}(t)=\epsilon(t+\frac{\tau}{2})-\epsilon(t-\frac{\tau}{2}) \label{pythagorean}
\end{equation}

又如符号函数 $sgn(t)$ ,也可以用单位阶跃函数表示为
\begin{equation}
sgn(t)=2\epsilon(t)-1
\end{equation}

单位冲激信号 $\delta(t)$ 的定义为
\begin{equation}
\delta(t)=\lim_{\tau \to 0}\frac{1}{\tau}G_{\tau}(t)
\end{equation}

单位冲激信号有如下特性
\begin{itemize}
\item 抽样(筛选)特性

若 $f(t)$ 连续且有界,则
\begin{equation}
\int_{-\infty}^{+\infty} {f(t)\delta(t-t_0)} \,{\rm d}t=f(t_0)
\end{equation}

\item $\delta(t)$ 为偶函数

\item $\delta(at)=\frac{1}{\left | a \right |}\delta(t)$
\end{itemize}

\subsection{连续时间系统的响应}
可以证明,一个线性时不变(LTI)系统可以用一个线性常系数微分方程表示,如式(15),这里举了一个二阶微分方程的例子。
\begin{equation}
\frac{{\rm d}^2}{{\rm d}t^2}r(t)+a_1\frac{{\rm d}}{{\rm d}t}r(t)+a_0r(t)=b_1\frac{{\rm d}}{{\rm d}t}e(t)+b_0e(t)
\end{equation}

这个方程的完全解由齐次解 $r_h(t)$ 和特解 $r_p(t)$ 组成(高等数学内容,此处复习一下)。齐次解又名自然响应,特解又名受迫响应。

以式(15)的微分方程为例,写出特征方程
\begin{equation}
\lambda^2+a_1\lambda+a_0=0 
\end{equation}

解得特征根 $\lambda_1$ 、 $\lambda_2$ 。

若 $\lambda_1\ne\lambda_2$ ,则方程的齐次解为
\begin{equation}
r_h(t)=C_1e^{\lambda_1t}+C_2e^{\lambda_2t}
\end{equation}

其中, $C_1$ 、 $C_2$ 为常数。若 $\lambda_1=\lambda_2$ ,则方程的齐次解为
\begin{equation}
r_h(t)=(C_1+C_2t)e^{\lambda_1t}
\end{equation}

当激励是常数时,特解也是一个常数。而当激励 $e(t)=e^{\alpha t}$ 时,特解应分三种情况讨论(我只列出来了一部分,完整版可以参考教材p32,因为\LaTeX 的表格,特别是需要合并单元格的表格,实在是太难画了,所以这里没展示出来):
\begin{itemize}
\item 当 $\alpha$ 不等于特征根时, $r_p(t)=Ae^{\alpha t}$ 。
\item 当 $\alpha$ 等于特征单根时, $r_p(t)=(A_1t+A_0)e^{\alpha t}$ 。
\item 当 $\alpha$ 等于2重特征根时, $r_p(t)=(A_2t^2+A_1t+A_0)e^{\alpha t}$ 。
\end{itemize} 

上述 $A$ 、 $A_0$ 、 $A_1$ 、 $A_2$ 皆为常数。需要特别说明的是,特解本身是就是微分方程的一个解。

讲到这里,有些同志可能还是不懂怎么通过这些知识求解响应,下面用一个例子说明。

\textbf{例1:}已知一系统的微分方程为 $\frac{{\rm d}}{{\rm d}t}r(t)+2r(t)=e(t)$ ,初始状态 $r(0_{-})=2$ ,求当激励为 $\epsilon(t)$ 时,系统的全响应。

\textbf{解:}由特征根方程 $\lambda +2=0$ 解得 $\lambda =-2$ ,因此可以得到齐次解(自然响应)
\begin{equation}
r_h(t)=Ce^{-2t}
\notag
\end{equation}

因为激励为常数,因此设特解(受迫响应)
\begin{equation}
r_p(t)=B
\notag
\end{equation}

由于特解本身就是微分方程的解,因此直接代入微分方程,解得 $B=0.5$ ,所以得到全响应
\begin{equation}
r(t)=Ce^{-2t}+\frac{1}{2}
\notag
\end{equation}

将初始状态 $r(0_{-})=2$ 代入全响应公式,解得 $C=1.5$ ,因此得到全响应
\begin{equation}
r(t)=\frac{3}{2}e^{-2t}+\frac{1}{2}\quad(t>0)
\notag
\end{equation}

现在,我们得到了求LTI响应的第一种方法:经典法。其步骤如下:
\begin{itemize}
\item 利用高等数学的传统方法写出带参数的齐次解和通解。
\item 将齐次解代入微分方程解出齐次解。
\item 将初始状态代入全响应解出全响应。
\end{itemize} 

而全响应也有另一种分割方法,即瞬态响应和稳态响应。瞬态响应指 $t\to \infty$ 时响应趋于零的那部分分量,暂态响应指 $t\to \infty$ 时响应不为零零的那部分分量。在例1中,瞬态响应为 $1.5e^{-2t}$ ,稳态响应为 $0.5$ 。需要特别指明的是,瞬态响应和稳态响应是无法直接求解出来的,只能在全响应求出来后进行分割。

\subsection{卷积}
卷积的定义为:
\begin{equation}
f_1(t)*f_2(t)=\int_{-\infty}^{+\infty} {f_1(\tau)f_2(t-\tau)} \,{\rm d}\tau
\end{equation}

卷积的一些性质:
\begin{itemize}
\item 交换律
\begin{equation}
f_1(t)*f_2(t)=f_2(t)*f_1(t)
\end{equation}

\item 分配率
\begin{equation}
f_1(t)*[f_2(t)+f_3(t)]=f_1(t)*f_2(t)+f_1(t)*f_3(t)
\end{equation}

\item 结合率
\begin{equation}
f_1(t)*[f_2(t)*f_3(t)]=[f_1(t)*f_2(t)]*f_3(t)
\end{equation}

\item 时移

若 $f_1(t)*f_2(t)=f(t)$ ,则有 $f_1(t-t_1)*f_2(t-t_2)=f(t-t_1-t_2)$ 。

\end{itemize}

一些常用信号的卷积(我只列出了我认为重要的一部分,更多常用信号的卷积可参考教材p41):
\begin{equation}
f(t)*\delta(t)=f(t)
\end{equation}
\begin{equation}
f(t)*\delta'(t)=f'(t)
\end{equation}
\begin{equation}
f(t)*\epsilon(t)=\int_{-\infty}^{t} {f(\tau)} \,{\rm d}\tau
\end{equation}
\begin{equation}
\frac{{\rm d}f(t)}{{\rm d}t}*\int_{-\infty}^{t} {g(\tau)} \,{\rm d}\tau=f(t)*g(t)
\end{equation}

\subsection{零输入响应与零状态响应}
系统的全响应还可以划分为零输入响应和零状态响应。零输入响应是输入激励为零时,仅由初始状态引起的响应,用 $r_{zi}(t)$ 表示。零状态响应是初始状态为零时,仅由激励引起的响应,用 $r_{zs}(t)$ 表示。

由定义可知,零输入响应的形式与齐次解是相同的,但是零输入响应是通过直接将初始状态代入解得的。

可以证明,一个LTI系统的零状态响应
\begin{equation}
r_{zs}(t)=h(t)*e(t)
\end{equation}

其中, $h(t)$ 是系统的冲激响应。

有了这些信息,我们可以将例1用另一种方法再做一遍。

\textbf{解:}由特征根方程 $\lambda +2=0$ 解得 $\lambda =-2$ ,因此可以得到零输入响应
\begin{equation}
r_{zi}(t)=Ae^{-2t}
\notag
\end{equation}

将初始状态 $r(0_{-})=2$ 代入上式,解得 $A=2$ ,因此解得零输入响应
\begin{equation}
r_{zi}(t)=2e^{-2t}
\notag
\end{equation}

因为零输入响应和齐次解的系数不一样,所以 $h(t)$ 必须带上 $e^{-2t}$ 项。因为 $h(t)$ 需要满足 $\frac{{\rm d}}{{\rm d}t}h(t)+2h(t)=\delta(t)$ ,所以 $h(t)$ 只能有 $e^{-2t}$ 项。因此设
\begin{equation}
h(t)=Be^{-2t}
\notag
\end{equation}

代入微分方程,解得 $B=1$ ,所以零状态响应(这一步使用了式(25))
\begin{equation}
r_{zs}(t)=e^{-2t}*\epsilon(t)=-\frac{1}{2}e^{-2t}+C
\notag
\end{equation}

写出全响应公式
\begin{equation}
r(t)=r_{zi}(t)+r_{zs}(t)=\frac{3}{2}e^{-2t}+C
\notag
\end{equation}

代入初始状态 $r(0_{-})=2$ ,解得全响应
\begin{equation}
r(t)=r_{zi}(t)+r_{zs}(t)=\frac{3}{2}e^{-2t}+\frac{1}{2}
\notag
\end{equation}

由此,我们得到了求LTI响应的第二种方法:时域零输入-零状态法。其步骤如下:
\begin{itemize}
\item 利用特征根写出带参数的零输入响应(其形式与齐次解相同)。
\item 将初始状态代入零输入响应,解得零输入响应。
\item 通过微分方程写出冲激响应的参数形式,然后代入微分方程解得冲激响应。
\item 通过激励与冲激响应的卷积写出零状态响应的参数形式。
\item 将初始状态代入全响应的参数形式,解得全响应。
\end{itemize} 


\end{document}